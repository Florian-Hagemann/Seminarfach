\documentclass[a4paper, 11pt, bibliography=totocnumbered]{scrartcl}

\usepackage[ngerman]{babel}
\usepackage{sans}
\usepackage[utf8]{inputenc}
\usepackage{csquotes}
\usepackage{tabularx}
\usepackage{setspace}
\onehalfspacing
\usepackage{geometry}

\usepackage[
  style=authoryear,
  backend=biber,
  language=ngerman,
  autocite=footnote
]{biblatex}

\renewcommand*{\postnotedelim}{\addcolon\space}
\DeclareFieldFormat{postnote}{#1}

\setlength{\bibitemsep}{\baselineskip}   % readable (recommended)
\addbibresource{expose.bib}

\title{Expose}
\author{Florian Hagemann}

\renewcommand{\labelenumii}{\arabic{enumi}.\arabic{enumii}}
\renewcommand{\labelenumiii}{\arabic{enumi}.\arabic{enumii}.\arabic{enumiii}}
\renewcommand{\labelenumiv}{\arabic{enumi}.\arabic{enumii}.\arabic{enumiii}.\arabic{enumiv}}



\begin{document}
\begin{titlepage}
    
    {\noindent\large Seminarfach „Glaube und Gesellschaft: \\Religiöse Phänomene zwischen Säkularisierung und Kommerzialisierung“}
    
\vspace*{100px}

\begin{center}
    \LARGE{\textbf{Exposé zum Verfassen einer wissenschaftlichen Facharbeit mit dem vorläufigen Arbeitstitel "Politische Instrumentalisierung von Religion: Ein Vergleich zwischen der römischen Republik und der USA"}}
\end{center} \vspace*{\fill}  
\textbf{Thema der Arbeit:} Politische Instrumentalisierung von Religion: Ein Vergleich zwischen der römischen Republik und der USA \\
\textbf{Verfasser:in:} Florian Hagemann \\
\textbf{Fachlehrkraft und Kursbezeichnung:} Frau Ulm-Wegner SF4 \vspace*{10px} \\
\textbf{Gymnasium Mellendorf} \\
\textbf{Fritz-Sennheiser-Platz 2} \\
\textbf{30900 Wedemark }\\
\textbf{Schuljahr: 2025/26} \\
\thispagestyle{empty}
\end{titlepage}

\section*{\begin{center}
    Seminarfach
\end{center}}

\textbf{Verfasser:in:} Florian Hagemann \\
\textbf{Thema:} Politische Instrumentalisierung von Religion: Ein Vergleich zwischen der römischen Republik und der USA \\
\textbf{Fachlehrkraft und Kursbezeichnung:} Frau Ulm-Wegner SF4 \\

\begin{center}
    \textbf{Abgabetermin: 19.02.2026 zu Beginn der Seminarfachsitzung um 14 Uhr}
\end{center}
\vspace*{50px}
\subsubsection*{Benotung}

\qquad Verfasser:in

\begingroup
    \vspace*{1\baselineskip}%
    \setlength{\tabcolsep}{0pt}%
    \setlength{\arrayrulewidth}{1.2pt}%
    \renewcommand*\arraystretch{1.5}%
    \sffamily
    \noindent%
    \begin{tabularx}{\linewidth}{
            >{\hspace{12pt}}X 
            c
            @{\hspace{3em}}
            >{\hspace{12pt}}X 
        }
        &&\\[\normalbaselineskip]
        \cline{1-1}\cline{3-3}
        Datum && Unterschrift der Lehrkraft
    \end{tabularx}
    \vspace*{1\baselineskip}
    \par
\endgroup

\pagestyle{empty}
\newpage

\tableofcontents
\thispagestyle{empty}
\setcounter{page}{0}
\newpage
\pagestyle{plain}
\newgeometry{left=3cm, right=5cm, top=1.5cm, bottom=2cm}






% Inhalt des Exposé
\section{Thema}
Die Trennung von Religion und Staat gilt als ein zentrales Prinzip moderner
Verfassungsstaaten und ist in den Vereinigten Staaten im ersten Zusatzartikel
zur Verfassung normiert\autocite[vgl.][1. Zusatzartikel]{USConst}. 
Dennoch spielt Religion weiterhin eine sichtbare Rolle im politischen Diskurs
der USA. Insbesondere in gesellschaftspolitischen Debatten, etwa im Kontext
der Abtreibungsfrage oder der Rechte von LGBTQ+-Personen, werden religiöse
Argumentationsmuster regelmäßig zur Legitimation politischer Positionen
herangezogen. Die enge Verflechtung protestantischer Milieus mit politischen
Eliten wurde dabei wiederholt untersucht\autocite[vgl.][23--30]{Schaefer2021}.
Auch aktuelle Analysen verweisen auf eine Verschiebung religiöser Autorität
im politischen Raum\autocite[vgl.][]{Hoover2021}.

Die politische Instrumentalisierung religiöser Praktiken ist jedoch kein
ausschließlich modernes Phänomen. Bereits in der römischen Republik war
Religion integraler Bestandteil des politischen Systems. Religiöse Ämter
waren eng mit politischer Macht verbunden, und kultische Handlungen konnten
konkrete politische Prozesse beeinflussen. Ein prominentes Beispiel ist das
Konsulat des Jahres 59 v. Chr., in dem Marcus Calpurnius Bibulus versuchte,
durch Berufung auf ungünstige Auspizien Gesetzesvorhaben seines Amtskollegen
Gaius Iulius Caesar zu blockieren. Die auguralen Praktiken boten dabei nicht
nur religiöse Orientierung, sondern eröffneten auch strategische
Handlungsspielräume innerhalb politischer Konflikte\autocite[vgl.][127--128]{DriedigerAugury2019}.

Ein weiteres Beispiel stellt die Blockade von Wahlen in den Jahren 57–56 v. Chr.
dar, bei der religiöse Argumentationen zur Verzögerung politischer Prozesse
eingesetzt wurden\autocite[vgl.][127--128]{DriedigerAugury2019}.

Vor diesem Hintergrund untersucht die vorliegende Arbeit die politische
Instrumentalisierung von Religion in der römischen Republik und in den
Vereinigten Staaten im Vergleich.

\section{Leitfrage}
Ausgehend von diesen Beobachtungen stellt sich die Frage, inwiefern sich
strukturelle Parallelen in der politischen Instrumentalisierung von Religion
zwischen der römischen Republik und den Vereinigten Staaten feststellen lassen.
Untersucht werden insbesondere die Motive der handelnden Akteure, die
konkreten Instrumentalisierungsmechanismen sowie die politischen und
gesellschaftlichen Folgen.

\section{Relevanz}
Das Verhältnis von Religion und Politik gehört zu den grundlegenden
Fragen politischer Ordnungen – sowohl in historischer als auch in
gegenwärtiger Perspektive. Obwohl moderne Verfassungsstaaten wie die
Vereinigten Staaten die institutionelle Trennung von Religion und Staat
im ersten Zusatzartikel zur Verfassung festschreiben\autocite[vgl.][Amendment I]{USConst},
bleibt Religion ein wirkmächtiger Faktor politischer Legitimation und
gesellschaftlicher Mobilisierung\autocite[vgl.][23--30]{Schaefer2021}.

Aktuelle gesellschaftspolitische Debatten zeigen, dass religiöse
Argumentationsmuster weiterhin zur Begründung politischer Positionen
herangezogen werden. Die Analyse dieser Prozesse ist daher nicht nur
historisch-deskriptiv, sondern trägt zum Verständnis gegenwärtiger
Konfliktlinien bei\autocite[vgl.][]{Hoover2021}.

Der Vergleich mit der römischen Republik eröffnet darüber hinaus eine
längsschnittliche Perspektive. In Rom war Religion kein vom Staat
getrennter Bereich, sondern integraler Bestandteil politischer
Institutionen. Religiöse Praktiken wie das Augurwesen konnten
konkrete politische Entscheidungsprozesse beeinflussen
\autocite[vgl.][127--128]{DriedigerAugury2019}.
Die Gegenüberstellung eines vormodernen und eines modernen
politischen Systems ermöglicht es, strukturelle Gemeinsamkeiten und
Unterschiede herauszuarbeiten und so allgemeine Muster politischer
Instrumentalisierung von Religion zu identifizieren.

Die Untersuchung ist daher sowohl für das historische Verständnis
politisch-religiöser Ordnungen als auch für die Einordnung aktueller
politischer Entwicklungen von Bedeutung.

\section{Ziel der Arbeit}
Ziel der Arbeit ist es, die Funktion religiöser Argumentations- und
Legitimationsmuster im politischen Raum vergleichend zu analysieren.
Dabei soll herausgearbeitet werden, ob Religion in beiden Systemen
als strategische Ressource zur Machtstabilisierung und
Konfliktsteuerung fungierte beziehungsweise fungiert.

Darüber hinaus soll geprüft werden, inwiefern sich trotz erheblicher
zeitlicher und struktureller Unterschiede vergleichbare Muster der
Instrumentalisierung erkennen lassen oder ob die jeweiligen
politischen Rahmenbedingungen entscheidend für Form und Wirkung
religiöser Einflussnahme sind.

\section{Grobe Gliederung}
\begin{enumerate}
    \item Einleitung
    \item Religion und Politik in der römischen Republik
    \begin{enumerate}
        \item Religiöse Ämter und politische Ordnung
        \item Das Augurwesen als politisches Instrument
        \item Fallbeispiele: Bibulus und die Wahlblockaden 57--56 v. Chr.
    \end{enumerate}
    \item Religion und Politik in den Vereinigten Staaten
    \begin{enumerate}
        \item Verfassungsrechtliche Grundlagen \autocite[vgl.][Amendment I.]{USConst}
        \item Protestantische Eliten und politische Mobilisierung\autocite[vgl.][]{Schaefer2021}
        \item Religiöse Argumentationen in aktuellen gesellschaftspolitischen Debatten\autocite[vgl.][]{Hoover2021}
    \end{enumerate}
    \item Vergleichende Analyse
    \begin{enumerate}
        \item Motive
        \item Strategien
        \item Politische Effekte
    \end{enumerate}
\end{enumerate}

\section{Methodik}






\pagebreak
\nocite{*}
\printbibliography

% formalia formalia
\restoregeometry
\newpage
\thispagestyle{empty}
\section*{Erklärung der Verfasser*innen}
Hiermit erklären ich, dass ich die vorliegende Arbeit selbständig angefertigt, keine anderen als die angegebenen Hilfsmittel benutzt und die Stellen der Facharbeit, die im Wortlaut oder im wesentlichen Inhalt aus anderen Werken oder dem Internet entnommen wurden, mit genauer Quellenangabe kenntlich gemacht habe. \vspace*{30px} \\
Verfasser*in: Florian Hagemann

\begingroup
    \vspace*{1\baselineskip}%
    \setlength{\tabcolsep}{0pt}%
    \setlength{\arrayrulewidth}{1.2pt}%
    \renewcommand*\arraystretch{1.5}%
    \sffamily
    \noindent%
    \begin{tabularx}{\linewidth}{
            >{\hspace{12pt}}X 
            c
            @{\hspace{3em}}
            >{\hspace{12pt}}X 
        }
        &&\\[\normalbaselineskip]
        \cline{1-1}\cline{3-3}
        Ort, Datum && handschriftliche Unterschrift
    \end{tabularx}
    \vspace*{1\baselineskip}
    \par
\endgroup

\end{document}