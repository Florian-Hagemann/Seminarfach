\documentclass[a4paper, 11pt, bibliography=totocnumbered]{scrartcl}

\usepackage[ngerman]{babel}
\usepackage{sans}
\usepackage[utf8]{inputenc}
\usepackage{csquotes}
\usepackage{tabularx}
\usepackage{setspace}
\onehalfspacing
\usepackage{geometry}

\usepackage[
  style=authoryear,
  backend=biber,
  language=ngerman
]{biblatex}

\setlength{\bibitemsep}{\baselineskip}   % readable (recommended)
\addbibresource{expose.bib}

\title{Expose}
\author{Florian Hagemann}

\begin{document}
\begin{titlepage}
    
    {\noindent\large Seminarfach „Glaube und Gesellschaft: \\Religiöse Phänomene zwischen Säkularisierung und Kommerzialisierung“}
    
\vspace*{100px}

\begin{center}
    \LARGE{\textbf{Exposé zum Verfassen einer wissenschaftlichen Facharbeit mit dem vorläufigen Arbeitstitel "Politische Instrumentalisierung von Religion: Ein Vergleich zwischen der römischen Republik und der USA"}}
\end{center} \vspace*{\fill}  
\textbf{Thema der Arbeit:} Politische Instrumentalisierung von Religion: Ein Vergleich zwischen der römischen Republik und der USA \\
\textbf{Verfasser:in:} Florian Hagemann \\
\textbf{Fachlehrkraft und Kursbezeichnung:} Frau Ulm-Wegner SF4 \vspace*{10px} \\
\textbf{Gymnasium Mellendorf} \\
\textbf{Fritz-Sennheiser-Platz 2} \\
\textbf{30900 Wedemark }\\
\textbf{Schuljahr: 2025/26} \\
\thispagestyle{empty}
\end{titlepage}

\section*{\begin{center}
    Seminarfach
\end{center}}

\textbf{Verfasser:in:} Florian Hagemann \\
\textbf{Thema:} Politische Instrumentalisierung von Religion: Ein Vergleich zwischen der römischen Republik und der USA \\
\textbf{Fachlehrkraft und Kursbezeichnung:} Frau Ulm-Wegner SF4 \\

\begin{center}
    \textbf{Abgabetermin: 19.02.2026 zu Beginn der Seminarfachsitzung um 14 Uhr}
\end{center}
\vspace*{50px}
\subsubsection*{Benotung}

\qquad Verfasser:in

\begingroup
    \vspace*{1\baselineskip}%
    \setlength{\tabcolsep}{0pt}%
    \setlength{\arrayrulewidth}{1.2pt}%
    \renewcommand*\arraystretch{1.5}%
    \sffamily
    \noindent%
    \begin{tabularx}{\linewidth}{
            >{\hspace{12pt}}X 
            c
            @{\hspace{3em}}
            >{\hspace{12pt}}X 
        }
        &&\\[\normalbaselineskip]
        \cline{1-1}\cline{3-3}
        Datum && Unterschrift der Lehrkraft
    \end{tabularx}
    \vspace*{1\baselineskip}
    \par
\endgroup

\pagestyle{empty}
\newpage

\tableofcontents
\thispagestyle{empty}
\setcounter{page}{0}
\newpage
\pagestyle{plain}
\newgeometry{left=3cm, right=5cm, top=1.5cm, bottom=2cm}

\section{Thema}
\section{Leitfrage}
\section{Relevanz}
\section{Ziel der Arbeit}
\section{Grobe Gliederung}

\pagebreak
\nocite{*}
\printbibliography

% formalia formalia
\restoregeometry
\newpage
\thispagestyle{empty}
\section*{Erklärung der Verfasser*innen}
Hiermit erklären ich, dass ich die vorliegende Arbeit selbständig angefertigt, keine anderen als die angegebenen Hilfsmittel benutzt und die Stellen der Facharbeit, die im Wortlaut oder im wesentlichen Inhalt aus anderen Werken oder dem Internet entnommen wurden, mit genauer Quellenangabe kenntlich gemacht habe. \vspace*{30px} \\
Verfasser*in: Florian Hagemann

\begingroup
    \vspace*{1\baselineskip}%
    \setlength{\tabcolsep}{0pt}%
    \setlength{\arrayrulewidth}{1.2pt}%
    \renewcommand*\arraystretch{1.5}%
    \sffamily
    \noindent%
    \begin{tabularx}{\linewidth}{
            >{\hspace{12pt}}X 
            c
            @{\hspace{3em}}
            >{\hspace{12pt}}X 
        }
        &&\\[\normalbaselineskip]
        \cline{1-1}\cline{3-3}
        Ort, Datum && handschriftliche Unterschrift
    \end{tabularx}
    \vspace*{1\baselineskip}
    \par
\endgroup

\end{document}