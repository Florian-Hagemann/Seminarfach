\documentclass[a4paper, 11pt, bibliography=totocnumbered]{scrartcl}

\usepackage[ngerman]{babel}
\usepackage{sans}
\usepackage[utf8]{inputenc}
\usepackage{csquotes}
\usepackage{tabularx}
\usepackage{setspace}
\onehalfspacing
\usepackage{geometry}

\usepackage[
  style=authoryear,
  backend=biber,
  language=ngerman,
  autocite=footnote
]{biblatex}

\renewcommand*{\postnotedelim}{\addcolon\space}
\DeclareFieldFormat{postnote}{#1}

\setlength{\bibitemsep}{\baselineskip}   % readable (recommended)
\addbibresource{expose.bib}

\title{Expose}
\author{Florian Hagemann}

\begin{document}
\begin{titlepage}
    
    {\noindent\large Seminarfach „Glaube und Gesellschaft: \\Religiöse Phänomene zwischen Säkularisierung und Kommerzialisierung“}
    
\vspace*{100px}

\begin{center}
    \LARGE{\textbf{Exposé zum Verfassen einer wissenschaftlichen Facharbeit mit dem vorläufigen Arbeitstitel "Politische Instrumentalisierung von Religion: Ein Vergleich zwischen der römischen Republik und der USA"}}
\end{center} \vspace*{\fill}  
\textbf{Thema der Arbeit:} Politische Instrumentalisierung von Religion: Ein Vergleich zwischen der römischen Republik und der USA \\
\textbf{Verfasser:in:} Florian Hagemann \\
\textbf{Fachlehrkraft und Kursbezeichnung:} Frau Ulm-Wegner SF4 \vspace*{10px} \\
\textbf{Gymnasium Mellendorf} \\
\textbf{Fritz-Sennheiser-Platz 2} \\
\textbf{30900 Wedemark }\\
\textbf{Schuljahr: 2025/26} \\
\thispagestyle{empty}
\end{titlepage}

\section*{\begin{center}
    Seminarfach
\end{center}}

\textbf{Verfasser:in:} Florian Hagemann \\
\textbf{Thema:} Politische Instrumentalisierung von Religion: Ein Vergleich zwischen der römischen Republik und der USA \\
\textbf{Fachlehrkraft und Kursbezeichnung:} Frau Ulm-Wegner SF4 \\

\begin{center}
    \textbf{Abgabetermin: 19.02.2026 zu Beginn der Seminarfachsitzung um 14 Uhr}
\end{center}
\vspace*{50px}
\subsubsection*{Benotung}

\qquad Verfasser:in

\begingroup
    \vspace*{1\baselineskip}%
    \setlength{\tabcolsep}{0pt}%
    \setlength{\arrayrulewidth}{1.2pt}%
    \renewcommand*\arraystretch{1.5}%
    \sffamily
    \noindent%
    \begin{tabularx}{\linewidth}{
            >{\hspace{12pt}}X 
            c
            @{\hspace{3em}}
            >{\hspace{12pt}}X 
        }
        &&\\[\normalbaselineskip]
        \cline{1-1}\cline{3-3}
        Datum && Unterschrift der Lehrkraft
    \end{tabularx}
    \vspace*{1\baselineskip}
    \par
\endgroup

\pagestyle{empty}
\newpage

\tableofcontents
\thispagestyle{empty}
\setcounter{page}{0}
\newpage
\pagestyle{plain}
\newgeometry{left=3cm, right=5cm, top=1.5cm, bottom=2cm}






% Inhalt des Exposé
\section{Thema}
Die Trennung von Kirche und Staat ist ein Prinzip stammend aus der Aufklärung und fest
in der Verfassung der Vereinigten Staaten von Amerika verankert \autocite[1. Zusatzartikel]{USConst}. Jedoch ist die Religion
nicht völlig aus der Politik verschwunden, sondern erfährt gerade eine Renaissance.
Mit der zweiten Amtzeit von Trump finden wir vielerlei Legislationsänderung im Name des
christlichen Glaubens; Die Abschaffung von Roe v. Wade und die Streichung von Unterstützung
von queeren Menschen sind zwei aktuelle Beispiele.

Doch dies ist nicht das erste Mal in der Geschichte, dass Religion benutzt wird um Gesetze
zu blockieren oder einzuführen. Auch schon in der römischen Republik wurde Religion instrumentalisiert
in der Politik. Ein Fallbeispiel ist dabei das Konsulat von Gaius Julius Caesar und
Marcus Calpurnius Bibulus im Jahr 59 v. Chr., wo Bibulus durch falsche Vogelschauten versuchte Caesars Gesetzentwürfe zu verhindern \autocite[127]{DriedigerAugury2019}.





\section{Leitfrage}


\section{Relevanz}


\section{Ziel der Arbeit}


\section{Grobe Gliederung}








\pagebreak
\nocite{*}
\printbibliography

% formalia formalia
\restoregeometry
\newpage
\thispagestyle{empty}
\section*{Erklärung der Verfasser*innen}
Hiermit erklären ich, dass ich die vorliegende Arbeit selbständig angefertigt, keine anderen als die angegebenen Hilfsmittel benutzt und die Stellen der Facharbeit, die im Wortlaut oder im wesentlichen Inhalt aus anderen Werken oder dem Internet entnommen wurden, mit genauer Quellenangabe kenntlich gemacht habe. \vspace*{30px} \\
Verfasser*in: Florian Hagemann

\begingroup
    \vspace*{1\baselineskip}%
    \setlength{\tabcolsep}{0pt}%
    \setlength{\arrayrulewidth}{1.2pt}%
    \renewcommand*\arraystretch{1.5}%
    \sffamily
    \noindent%
    \begin{tabularx}{\linewidth}{
            >{\hspace{12pt}}X 
            c
            @{\hspace{3em}}
            >{\hspace{12pt}}X 
        }
        &&\\[\normalbaselineskip]
        \cline{1-1}\cline{3-3}
        Ort, Datum && handschriftliche Unterschrift
    \end{tabularx}
    \vspace*{1\baselineskip}
    \par
\endgroup

\end{document}